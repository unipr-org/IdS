\documentclass[12pt,a4paper]{article}
\usepackage{hyperref}
\hypersetup{
  colorlinks=true,
  linkcolor=blue,
  urlcolor=cyan,
}
\urlstyle{same}
% \usepackage[condensed,math]{kurier}
% \usepackage[T1]{fontenc}
\usepackage{svg}
\usepackage{tikz}
\usepackage{stanli}
\usepackage{afterpage}
\usepackage{multirow}
\usepackage{subfig}
\usepackage{pgfpages}
\usepackage{svg}
\usepackage{rotating}

\usepackage{listings}
\usepackage{color}

\definecolor{dkgreen}{rgb}{0,0.6,0}
\definecolor{gray}{rgb}{0.5,0.5,0.5}
\definecolor{mauve}{rgb}{0.58,0,0.82}

\lstset{frame=tb,
  language=Java,
  aboveskip=3mm,
  belowskip=3mm,
  showstringspaces=false,
  columns=flexible,
  basicstyle={\small\ttfamily},
  numbers=none,
  numberstyle=\tiny\color{gray},
  keywordstyle=\color{blue},
  commentstyle=\color{dkgreen},
  stringstyle=\color{mauve},
  breaklines=true,
  breakatwhitespace=true,
  tabsize=4
}

% Language setting
% Replace `english' with e.g. `spanish' to change the document language
\usepackage[italian]{babel}

% Set page size and margins
% Replace `letterpaper' with `a4paper' for UK/EU standard size
\usepackage[a4paper,top=2cm,bottom=1.5cm,left=1.5cm,right=1.5cm,marginparwidth=1.75cm]{geometry}

% Useful packages
\usepackage{amsmath}
\usepackage{graphicx}

\usepackage[lastexercise,answerdelayed]{exercise}
\renewcounter{Exercise}[subsection]
\counterwithin{Exercise}{subsection}
\renewcommand{\ExerciseName}{Esercizio}

\title{}
\author{}
\date{}

\begin{document}
	
	\newcommand{\subf}[2]{%
		{\small\begin{tabular}[t]{@{}c@{}}
				#1\\#2
		\end{tabular}}%
	}
	
	\begin{titlepage}
		\begin{center}
			\vspace*{3cm}
			
			\Huge
			\textbf{Esercizi di Ingegneria del software}
			
			\vspace{0.3cm}
			\Huge
			2023/2024
			
			\vspace{0.8cm}
			\large
			
			%INSTRUCTED BY: MRS. A.A.S.KAUSHLYA
			
			
			\vspace{0.5cm}
			\LARGE
			
			
			\vspace{1.5cm}
			
			\textbf{}
            \includegraphics[width=0.4\textwidth]{unipr_logo.jpg}
			
			\vfill
			
			
			
			\vspace{0.8cm}
			
			
			
			\Large
			
			
			
			
		\end{center}
		\Large
		\begin{tabbing}
			\hspace*{1em}\= \hspace*{8em} \= \kill % set the tabbings
			\> Nome:\>  \textbf{Di Agostino Manuel} \\
			\> Insegnamento:\>  Ingegneria del software  \\
			\> Anno:  \> 2023/2024
		\end{tabbing}
		
	\end{titlepage}
	\clearpage
	
	
\section{Design patterns}
    \subsection{Composite pattern}\label{patter:composite}
    
    \begin{Exercise}\label{composite:ex1}
    Si vuole modellare la mappa di un videogioco \textit{open-world} utilizzando una struttura dati ad albero; a seconda del livello di zoom dell'interfaccia è infatti possibile osservare una versione più dettagliata della mappa, corrispondente ad un particolare nodo dell'albero.
    Esistono tre categorie di nodi:
    \begin{itemize}
        \item \textbf{foglia}: non ulteriormente espandibile, corrisponde ad un preciso punto fisso sulla mappa, importante per la logica del gioco;
        \item \textbf{distretto}: identifica zone più ampie, ad esempio un quartiere cittadino formato dalla composizione di nodi foglie;
        \item \textbf{regione}: aggrega più distretti contigui;
        \item \textbf{stato}: il livello più alto nella gerarchia considerata.
    \end{itemize}
    Ad ogni nodo sono inoltre sempre associati un \textit{nome} e una \textit{tipologia} conforme alla seguente interfaccia:

    \begin{lstlisting}
        public enum NodeType {
            LEAF,
            DISTRICT,
            REGION,
            COUNTRY
        }
    \end{lstlisting}
    
    Realizzare un class diagram UML che descriva la struttura della soluzione e fornire un'implementazione in Java che ne soddisfi i requisiti.
    \end{Exercise}

    %% END COMPOSITE SECTION
    \subsection{Decorator pattern}
    \begin{Exercise}[origin={Ispirato ad un esempio del libro GoF}]
        I flussi sono un'astrazione fondamentale nella maggior parte delle strutture di I/O. Un flusso può fornire un'interfaccia per convertire oggetti in una sequenza di byte o caratteri. Questo ci consente di trascrivere un oggetto su un file o su una stringa in memoria per il recupero successivo. Un modo diretto per farlo è definire una classe astratta \texttt{Stream} con le sottoclassi \texttt{MemoryStream} e \texttt{FileStream}.
        Il corpo della classe \texttt{Stream} è il seguente:
        \begin{lstlisting}
            public abstract class Stream {
                // private data section
                
                public void PutInt() {
                    // impl
                }
                public void PutString() {
                    // impl
                }
                public abstract void HandleBufferFull();
            }
        \end{lstlisting}
        Le classi \texttt{MemoryStream} e \texttt{FileStream} si occupano di ridefinire il metodo astratto \texttt{HandleBufferFull()} per scrivere direttamente sulla memoria RAM e rispettivamente su file.
        
        Supponiamo che la classe astratta \texttt{Stream} manipoli il buffer di caratteri del flusso tramite codifica \texttt{UTF-8}; si aggiunga la funzionalità di conversione di codifica del testo in \texttt{ASCII} standard a \texttt{7 bit} senza intaccare le classi/interfacce sopra citate, utilizzando dunque il \textit{Decorator pattern}.

        Fornire anche un class diagram UML della soluzione.
    \end{Exercise}

    %% END DECORATOR SECTION
    \subsection{Command pattern}
    \begin{Exercise}
        Si vuole realizzare la logica applicativa di un editor di testo che permetta di manipolare file testuali in codifica \texttt{ASCII} standard a \texttt{7 bit}. È previsto che l'interfaccia \texttt{EditableFile} esponga una serie di funzionalità:
        \begin{itemize}
            \item \textbf{creazione} del file, a partire da un nome scelto dall'utente
            \item \textbf{eliminazione} del file
            \item \textbf{lettura} completa del file
            \item \textbf{lettura} parziale del file, specificando punto di inizio e fine (numeri riga)
            \item \textbf{concatenamento} di testo alla fine del file
            \item \textbf{modifica} parziale del file, specificando punto di inizio e fine (numeri riga)
            \item \textbf{ridenominazione} del file
            \item \textbf{salvataggio} del file
        \end{itemize}
        Bisogna inoltre prevedere la possibilità di annullare fino a \textit{256 modifiche} effettuate dall'utente.

        Scrivere un'implementazione in Java del sistema descritto fornendone una descrizione tramite class diagram UML.
    \end{Exercise}
    \begin{Exercise}
        Si consideri la seguente interfaccia Java che descrive una struttura dati \textit{queue}:
        \begin{lstlisting}
            public interface Queue<T> {
                public void enqueue(T elem);
                public T dequeue();
                public void clear();
                public boolean isEmpty();
            }
        \end{lstlisting}
        L'interfaccia in questione deve essere implementata dalla classe \textbf{SimpleQueue}, che offre un costruttore senza paramentri per la creazione di una coda vuota. La coda vuole permettere l'operazione di \textit{undo} per l'ultima modifica effettuata.
        Descrivere la classe descritta mediante un class diagram UML e fornirne un implementazione in Java. 
    \end{Exercise}


    %% END COMMAND SECTION
    \subsection{Visitor pattern}\label{pattern:visitors}
    \begin{Exercise}
    
    [Richiede l'\hyperref[composite:ex1]{esercizio \ref{composite:ex1}}]
    
    Sulla struttura dati implementata in \hyperref[composite:ex1]{\ref{composite:ex1}} sono richieste le seguenti azioni:
    \begin{itemize}
        \item possibilità di creare elenchi dei nodi suddivisi per categoria (\texttt{LEAF}, \texttt{DISTRICT}, \texttt{REGION}, \texttt{COUNTRY})
        \item visita in ampiezza per livelli, a partire da un determinato nodo e con una profondità massima, utile al motore di rendering della mappa
    \end{itemize}
    \end{Exercise}
	
\end{document}